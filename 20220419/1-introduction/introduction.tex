%
% SECTION: SECURITYMADEIN.LU
%
\section*{Who we are}
\begin{frame}
  \frametitle{SECURITYMADEIN.LU}
  Our history:
  \begin{center}
    \begin{itemize}
      \item 2003: Cyberworld Awareness and Security Enhancement Services (\textbf{CASES});
      \item 2007: Computer Incident Response Center Luxembourg (\textbf{CIRCL});
      \item 2010: SECURITYMADEIN.LU is a \textit{GIE} (Groupement d’Intérêt Économique);
      \item 2017: Cyber security Competence Center (\textbf{C3}).
    \end{itemize}
  \end{center}
  CASES is an initiative of the Ministry of Economy after the worm
  \textit{I love you} decimated more than 3 millions computers in less than a week.
\end{frame}

\begin{frame}
  \frametitle{CASES}
  \framesubtitle{}
  \begin{block}{Mission}
    Foster cyber security by supporting Luxembourg administrations and SMEs.
  \end{block}

  \begin{block}{Services}
    \begin{center}
      \begin{itemize}
        \item \textbf{Awareness}: publications of articles and videos;
        \item \textbf{Trainings}:
        introduction to cyber security for different audiences;
        \item \textbf{Software}:
        MONARC, MOSP, Fit4Cybersecurity, etc.
      \end{itemize}
    \end{center}
  \end{block}

  \begin{block}{Cooperations}
    ANSSI-LU,
    Centre for Cyber Security Belgium, KonzeptAcht GmbH, ILR, GRC-Luxembourg and others.
  \end{block}
\end{frame}

% --------- Summary ---------
\setcounter{tocdepth}{1}
\begin{frame}
  \frametitle{Content at glance}
  \tableofcontents
\end{frame}
\setcounter{tocdepth}{4}
% ----------------------------

%
% SECTION: What is MONARC?
%
\section{What is MONARC?}
\begin{frame}
  \frametitle{Summary}
  \tableofcontents[currentsection, hideothersubsections]
\end{frame}
\subsection{An open source software}
\begin{frame}
  \frametitle{An open source software}
  \framesubtitle{}
  MONARC is the tool you need for an optimised, precise and repeatable risk assessment.

  \bigskip
  \begin{itemize}
    \item Web application (SaaS, self-hosted, virtual machine, etc.);
    \item source code\footnote{\url{https://github.com/monarc-project}}:
    \texttt{GNU Affero General Public License version 3};
    \item data: \texttt{CC0 1.0 Universal - Public Domain Dedication}.
  \end{itemize}

  \bigskip
  MONARC is easy to use.

  Used and recognized by experts from different fields (not only information security).

  \bigskip
  For many users, it started with a spreadsheet!
\end{frame}

\subsection{A community}
\begin{frame}
  \frametitle{A community}
  \framesubtitle{}
  \begin{itemize}
    \item more than 260 organizations:\\ \url{https://my.monarc.lu};
    \item 17 organizations sharing MONARC objects (threats, assets, recommendations, etc.):\\
    \url{https://objects.monarc.lu};
    \item a global dashboard with trends about threats and vulnerabilitties:\\
    \url{https://dashboard.monarc.lu};
    \item discussions on GitHub:\\
    \url{https://github.com/monarc-project/MonarcAppFO/discussions}.
  \end{itemize}
\end{frame}

\subsection{A method}
\begin{frame}
  \frametitle{A method}
  \framesubtitle{Based on \texttt{ISO/IEC 27005:2011}, but optimized}
  \begin{center}
    \includegraphics[scale=0.6]{../common_pictures/iso27005-2011.png}
  \end{center}
\end{frame}
