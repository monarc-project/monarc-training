%
% SECTION: Security Made In Lëtzebuerg
%
\section*{Who we are}
\begin{frame}
    \frametitle{Security Made In Lëtzebuerg (SMILE)}
    Our timeline
    \begin{center}
        \begin{itemize}
            \item 2003: Cyberworld Awareness and Security Enhancements Services (\textbf{CASES});
            \item 2007: Computer Incident Response Center Luxembourg (\textbf{CIRCL});
            \item 2010: SMILE is a \textit{GIE} (Groupement d’Intérêt Économique);
            \item 2017: Cyber security Competence Center (\textbf{C3}).
        \end{itemize}
    \end{center}
\end{frame}

\begin{frame}
    \frametitle{CASES}
    \framesubtitle{}
    \begin{block}{Mission}
        Promote information security by supporting Luxembourg administrations and SMEs.
    \end{block}
    \bigskip
    Services:
    \begin{center}
        \begin{itemize}
            \item \textbf{awareness}: article publications;
            \item \textbf{trainings}:
                introduction to cyber security for different audiences \footnote{\url{https://www.cases.lu/services/trainings.html}};
            \item \textbf{software}:
                MONARC, Fit4Cybersecurity, MOSP, TACOS, etc.\footnote{\url{https://github.com/CASES-LU}}
        \end{itemize}
    \end{center}
\end{frame}

% --------- Summary ---------
\setcounter{tocdepth}{1}
\begin{frame}
    \frametitle{Content at glance}
    \tableofcontents
\end{frame}
\setcounter{tocdepth}{4}
% ----------------------------

%
% SECTION: What is MONARC?
%
\section{What is MONARC?}
\begin{frame}
    \frametitle{Summary}
    \tableofcontents[currentsection, hideothersubsections]
\end{frame}
\subsection{An open source software}
\begin{frame}
\frametitle{An open source software}
\framesubtitle{}
    \begin{itemize}
        \item Web application (SaaS, self-hosted, virtual machine, etc.);
        \item source code\footnote{\url{https://github.com/monarc-project}} under \texttt{GNU Affero General Public License version 3};
        \item data under \texttt{CC0 1.0 Universal (CC0 1.0) - Public Domain Dedication}.
    \end{itemize}
    \bigskip
    For many users, it started with a spreadsheet.
\end{frame}

\subsection{A community}
\begin{frame}
\frametitle{A community}
\framesubtitle{}
    \begin{itemize}
        \item sharing of risk models and all kind of objects (assets, threats, vulnerabilitties, recommendations, referentials, etc.);
        \item data available via a sharing platform: MOSP\footnote{\url{https://objects.monarc.lu/organization/MONARC}};
        \item more than 130 organizations on \url{https://my.monarc.lu}.
    \end{itemize}
\end{frame}

\subsection{A method}
\begin{frame}
\frametitle{A method}
\framesubtitle{Based on \texttt{ISO/IEC 27005:2011}, but optimized}
    \begin{center}
        \includegraphics[scale=0.6]{../common_pictures/iso27005-2011.png}
    \end{center}
\end{frame}
